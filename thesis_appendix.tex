%------------------------------------------------------------------------------
\chapter{Appendix}
\label{sec:app}
%------------------------------------------------------------------------------

\section{Solving the geodesic equation in a two dimensional metric}
\label{sec:app_congruence}
Let us first assume we already solved the geodesic equation over the whole two dimensional spacetime (call it \(\mathcal{M}\)). Then we can construct lightcone coordinates: Fix a point \(\vb{x}_0\) in the spacetime. In this point there exist two linear independent null vector namely \(\vb{t}\) and another one defined by \(\vb{n}^2 = 0\) and \(\vb{t}\cdot\vb{n} = -1\). Note that since we are in a two dimensional spacetime \(\vb{n}\) is uniquely defined. Then solve for the geodesic starting at \(\vb{x}_0\) with tangent vector \(\vb{n}\). Associate a point on the geodesic with the corresponding value of the affine parameter \(\lambda\). Then starting at such a point solve the geodesic (call it \(\Sigma\)) with tangent vector \(\vb{t}\) and associate every point on this geodesic with the value of the affine parameter \(\tau\). By this we can find a map \(\vb{x}(\tau,\lambda)\). We can use this map for a coordinate transformation which yields to the coordinate system \(\partial_\tau = \vb{t}\) and \(\partial_\lambda = \vb{n}\).

For the boundary condition take \(\vb{t}\vb{n} = -1\) on \(\Sigma\). Before we start let us summarize the known properties of \(\vb{t}\) and \(\vb{n}\): Clearly on the whole spacetime \(\nabla_{\vb{t}}\vb{t} = 0\) and \(\vb{t}^2 = 0\). Also since they are a coordinate system \([\vb{t},\vb{n}] = 0\) which means \(\covd{t}\vb{n} = \nabla_{\vb{n}}\vb{t}\). On \(\Sigma\) we also know that \(\vb{t}\vb{n} = -1\) and \(\vb{n}^2 = 0\).

First observe that
\begin{align}
\vb{t} \covd{t}\vb{n} = \vb{t} \covd{n}\vb{t} = \frac{1}{2} \covd{n}\vb{t}^2 = 0
\label{equ:congruence_tdtn}
\end{align}
in the whole spacetime. Use this to calculate \( \covd{t}(\vb{t}\vb{n}) = \vb{t}\covd{t}\vb{n} = 0\) which means that
\begin{align}
\vb{t}\vb{n} = \mathrm{const.} = -1 
\end{align}
This also implies that \(\covd{t}\vb{n} = -\vb{t} (\vb{n}\covd{t}\vb{n})\,\)(since \(\vb{t} \covd{t}\vb{n} = 0\) and therefore \(\covd{t}\vb{n} \sim \vb{t}\)).

Next derive
\begin{align}
\vb{n}\covd{n}\vb{t} = \covd{n}(\vb{t}\vb{n} - \vb{t}\covd{n}\vb{n} = \covd{n}(-1) - \vb{t}\covd{n}\vb{n}) = - \vb{t}\covd{n}\vb{n}
\end{align}

Note that on \(\Sigma\): \(\covd{n}\vb{n} = 0\) which means \(\vb{n}\covd{n}\vb{t} = 0\) and (by eq. \ref{equ:congruence_tdtn} )\(\covd{n}\vb{t} = \covd{t}\vb{n} = 0\). Unfortunately this parallel transport condition is only satisfied on \(\Sigma\) not on \(\mathcal{M}\). Therefore two things will happen to \(\vb{n}\): it won't remain a null vector and it will not solve the geodesic equation outside of \(\Sigma\). To see this calculate
\begin{align}
\covd{t}\covd{t}\vb{n}^2 &= 2\covd{t}(\vb{n}\covd{t}\vb{n}) = 2(\covd{t}\vb{n})^2 + 2 \vb{n}\covd{t}\covd{t}\vb{n}\\
	&\overset{\covd{t}\vb{n} \sim \vb{t}}{=} 2 \vb{n}\covd{t}\covd{n}\vb{t} = 2\vb{n} R(\vb{t},\vb{n})\vb{t} + 2 \vb{n}\covd{n}\covd{t}\vb{t}\\
	&\overset{\covd{t}\vb{t} = 0}{=} 2 \vb{n}\mathrm{R}(\vb{t},\vb{n})\vb{t}
\end{align}

where \(\mathrm{R}(\vb{a},\vb{b}) = \covd{a}\covd{b}-\covd{b}\covd{a} - \nabla_{[\vb{a},\vb{b}]}\) is the curvature tensor. If it vanishes \(\dv[2]{\vb{n}^2}{\tau} = 0\) and so \(\vb{n}^2 = a\tau + b\). But from the boundary condition on \(\Sigma\) follows that \(a = b = 0\) and so \(\vb{n}^2 = 0\). However when there is curvature (as in our case) \(\vb{n}^2\) will differ from \(0\). Since the behaviour of \(\vb{n}^2\) fully determines \(\covd{n}\vb{n}\)
\begin{align}
\vb{t}\covd{n}\vb{n} &= \covd{n}(\vb{t}\vb{n}) - \vb{n}\covd{n} \vb{t} = - \vb{n}\covd{t}\vb{n} = -\frac{1}{2} \covd{t}\vb{n}^2\\
\vb{n}\covd{n}\vb{n} &= \frac{1}{2} \covd{n}\vb{n}^2
\end{align}  

this also means that \(\covd{n}\vb{n} \neq 0\). However is the curvature is small one can neglect this change and so \(\vb{n}^2 \approx 0\) and \(\covd{n}\vb{n} \approx 0\). One may then keep track of two neighbouring geodesics by computing the null geodesic between them and evaluating it at the corresponding \(\lambda\) value. Frankly speaking this means that the (null geodesic) distance \(\lambda\) between null geodesics will remain constant.

\section{Calculating the scalarproduct between early and later modes}
\label{sec:app_scalarproduct}
We would like to calculate the scalarproduct between the following modes in the later spacetime:

\begin{align}
u_{\omega l m} &= \frac{i^{-l}}{2i\sqrt{\pi\omega}r} e^{-i\omega u} Y_l^m (\theta, \phi) - \frac{i^{l}}{2i\sqrt{\pi\omega}r} e^{-i\omega v} Y_l^m (\theta, \phi)\\
\psi_{\omega l m} &= \frac{i^{-l}}{2i\sqrt{\pi\omega}r} e^{i\omega B e^{-\frac{u}{4M}}} Y_l^m (\theta, \phi) - \frac{i^{l}}{2i\sqrt{\pi\omega}r} e^{-i\omega v} Y_l^m (\theta, \phi)
\end{align}

To do so we choose as hypersurface a lightlike surface with \(v = \mathrm{const}.\) together with some spacelike surface that captures the complete interior region (we need both surfaces since \(\Im^+\) is only a partial Cauchysurface \footnote{Note that \(v = \mathrm{const}.\) is a lightlike surface but partial Cauchysurfaces need to be spacelike. So it is not a partial Cauchysurface. However as eq. \ref{equ:qft_scalarproduct_invariant} is also true for lightlike surfaces the value of the scalarproduct will not change.}). However the the \(u\) modes are \(0\) inside the black hole and therefore the integral over the interior will vanish.

So we will only integrate over the lightlike surface. The normal vector of the surface is given by \(S = \frac{r^2 \sin\theta}{f(r)} \partial_u\). We will neglect the factor \(f(r) \approx 1\) since the bigger part of the hypersurface will be far away from the black hole and we assume that the non approximate wave functions will drop to zero at the event horizon\footnote{Actually Hawking and all other authors I encountered so far didn't mentioned this factor.}.

Before evaluating the integral let us rewrite the modes. First define the prefactor as \(A = \frac{i^{-l}}{2i\sqrt{\pi\omega}}\) and then  
\begin{align}
u_{\omega l m} &= \frac{\tilde{u}_\omega}{r} Y_l^m (\theta, \phi)\\
\psi_{\omega l m} &= \frac{\tilde{\psi}_\omega}{r} Y_l^m (\theta, \phi)\\
\tilde{u}_{\omega} &= A e^{-i\omega u} + A^* e^{-i\omega v}\\
\tilde{\psi}_{\omega} &= A e^{i\omega B e^{-\frac{u}{4M}}} + A^* e^{-i\omega v}
\end{align}

Using this we can simplify the scalarproduct (we will drop the indices \(l, m\) because the angular integral will just give \(\delta_{ll'}\delta_{mm'}\)):
\begin{align}
(u_{\omega'}|\psi_\omega) &= i\int_{-\infty}^\infty r^2 \dd{u} \frac{\tilde{u}_{\omega'}^*}{r} \partial_u \frac{\tilde{\psi}_\omega}{r} - \frac{\tilde{\psi}_\omega}{r} \partial_u \frac{\tilde{u}_{\omega'}^*}{r}\\
	&= i\int_{-\infty}^\infty r^2 \dd{u} \frac{\tilde{u}_{\omega'}^*}{r} \frac{\partial_u \tilde{\psi}_\omega}{r} - \frac{\tilde{\psi}_\omega}{r} \frac{\partial_u \tilde{u}_{\omega'}^*}{r} - \frac{\tilde{u}_{\omega'}^*}{r} \frac{\tilde{\psi}_\omega}{r^2} \partial_u r + \frac{\tilde{\psi}_\omega}{r} \frac{\tilde{u}_{\omega'}^*}{r^2} \partial_u r\\
	&= i\int_{-\infty}^\infty \dd{u} \tilde{u}_{\omega'}^* \partial_u \tilde{\psi}_\omega - \tilde{\psi}_\omega \partial_u \tilde{u}_{\omega'}^*\\
	&= -2i\int_{-\infty}^\infty \dd{u} \tilde{\psi}_\omega \partial_u \tilde{u}_{\omega'}^*
\end{align}

In the last step we integrated by parts and assume that the boundary terms vanish (We know that the later modes drop to zero at the horizon \(u = \infty\). For \(u = -\infty\) we have a rapidly oscillating function which is zero at average).

\begin{align}
(u_{\omega'}|\psi_\omega) &= -2i\int_{-\infty}^\infty \dd{u} \tilde{\psi}_\omega \partial_u \tilde{u}_{\omega'}^*\\
	&= 2\omega' \int_{-\infty}^\infty \dd{u} \qty(A e^{i\omega B e^{-\frac{u}{4M}}} + A^* e^{-i\omega v}) A'^* e^{i\omega' u}\\
	&= 2\omega' AA'^* \int_{-\infty}^\infty \dd{u} e^{i\omega B e^{-\frac{u}{4M}}} e^{i\omega' u} + 2\omega' A^*A'^* e^{-i\omega v} \delta(\omega')\\
	&= 2\omega' AA'^* \int_{-\infty}^\infty \dd{u} e^{i\omega B e^{-\frac{u}{4M}}} e^{i\omega' u}
\end{align}

Next substitute \(x = e^{-\frac{u}{4M}}\) and then use contour integration to integrate over the positive imaginary axis (\(x = i y\)):
\begin{align}
(u_{\omega'}|\psi_\omega) &= 8M \omega' A^*A'^* \int_{0}^\infty \frac{\dd{x}}{x} e^{i\omega B x} e^{-4M i \omega' \ln x}\\
&= 8M \omega' AA'^* \int_{0}^\infty \dd{x} e^{i\omega B x} x^{-4M i \omega' - 1}\\
&= i 8M \omega' AA'^* \int_{0}^\infty \dd{y} e^{-\omega B y} (iy)^{-4M i \omega' - 1}\\
&= i^{-4M i \omega'} 8M \omega' AA'^* \int_{0}^\infty \dd{y} e^{-\omega B y} y^{-4M i \omega' - 1}\\
&\overset{z = \omega B y}{=} i^{-4M i \omega'} 8M \omega' AA'^* \int_{0}^\infty \frac{\dd{z}}{B\omega} e^{-z} z^{-4M i \omega' - 1} (\omega B)^{4M i \omega' + 1}\\
&= i^{-4M i \omega'} (\omega B)^{4M i \omega'} 8M \omega' AA'^* \Gamma(-4M i \omega')
\label{equ:app_scalarproduct}
\end{align}

To calculate \((u_{\omega'}|\psi_\omega*)\) one can redo the same calculation but choose a contour over the negative imaginary axis (\(x = - iy\))

\begin{align}
(u_{\omega'}|\psi_\omega^*) &= -2i\int_{-\infty}^\infty \dd{u} \tilde{\psi}_\omega^* \partial_u \tilde{u}_{\omega'}^*\\
	&= 2\omega' A^*A'^* \int_{-\infty}^\infty \dd{u} e^{-i\omega B e^{-\frac{u}{4M}}} e^{i\omega' u}\\
	&= 8M \omega'A^*A'^* \int_{0}^\infty \dd{x} e^{-i\omega B x} x^{-4M i \omega' - 1}\\
	&= -i 8M \omega' A^*A'^* \int_{0}^\infty \dd{y} e^{-\omega B y} (-iy)^{-4M i \omega' - 1}\\
	&= i^{4M i \omega'} 8M \omega' A^*A'^* \int_{0}^\infty \dd{y} e^{-\omega B y} y^{-4M i \omega' - 1}\\
	&\overset{z = \omega B y}{=} i^{4M i \omega'} 8M \omega' A^*A'^* \int_{0}^\infty \frac{\dd{z}}{\omega B} e^{-z} z^{-4M i \omega' - 1} (\omega B)^{4M i \omega' + 1}\\
	&= i^{4M i \omega'} (\omega B)^{4M i \omega'}  8M \omega' A^*A'^* \Gamma(-4M i \omega')\\
	&= (-1)^{l+1} i^{8M i \omega'} (u_{\omega'}|\psi_\omega) = (-1)^{l+1} e^{-4\pi M \omega'} (u_{\omega'}|\psi_\omega) 
\end{align}

So both scalar products lead (up to a prefactor) to the same result.

In eq. \ref{equ:hawking_bb} it was shown that \(\bra*{0}b_{\tilde{\omega}} b_{\omega'}\ket*{0}\) vanishes for \(\tilde{\omega} \neq \omega'\). For \(\tilde{\omega} = \omega'\) we still need to evaluate the integral directly:

\begin{align}
\bra*{0}b_{\tilde{\omega}} b_{\omega'}\ket*{0} &= (-1)^{l+1} e^{-4 M\pi \omega'} \int_0^\infty \dd{\omega} (u_{\omega'}|\psi_\omega) (u_{\omega'}|\psi_\omega)\\
&= (-1)^{l+1} e^{-4 M\pi \omega'} \int_0^\infty \dd{\omega} i^{-8M i \omega'} (\omega B)^{8M i \omega'} (8M)^2 \omega'^2 A^2A'^{2*} \Gamma(-4M i \omega')^2\\
	&= B^{8M i \omega'} (8M)^2 \omega'^2 \frac{1}{4\pi} A'^{2*} \Gamma(-4M i \omega')^2 \int_0^\infty \dd{\omega}\omega^{8M i \omega'} \frac{1}{\omega}\\
	&= (8M)^2 B^{8M i \omega'} \omega'^2 \frac{1}{4\pi} A'^{2*} \Gamma(-4M i \omega')^2 \int_0^\infty \dd{\omega}e^{(8M i \omega' - 1)\ln \omega}\\
	&\overset{x = \ln\omega}{=} B^{8M i \omega'} (8M)^2 \omega'^2 \frac{1}{4\pi} A'^{2*} \Gamma(-4M i \omega')^2 \int_{-\infty}^\infty \dd{x} e^x e^{(8M i \omega' - 1)x}\\
	&= B^{8M i \omega'} (8M)^2 \omega'^2 \frac{1}{4\pi} A'^{2*} \Gamma(-4M i \omega')^2 \delta(8M\omega')\\
	&= B^{8M i \omega'} 8M \omega'^2 \frac{1}{4\pi} A'^{2*} \Gamma(-4M i \omega')^2 \delta(\omega')
\end{align}

There is only a contribution for \(\omega' = 0\) which we excluded from our analysis.
