%------------------------------------------------------------------------------
\chapter{Appendix}
\label{sec:app}
%------------------------------------------------------------------------------

\section{Wightmanfunction in Minkowskispace}
\todo{old notation}
Using eq. \ref{equ:wightman_modes} we can calculate the Wightman function:
\begin{align}
D^+(x,x') &= \int \mathrm{d}^3 k\, u_{\vec{k}}(x) u_{\vec{k}}^*(x')\\
	&= \int \frac{\mathrm{d}^3 k}{(2\pi)^3}\, \frac{1}{2|k|} e^{- i |k| (t-t') + i \vec{k} (\vec{x}-\vec{x}')}\\
	&\overset{\omega = |k|}{=} \int_0^\infty \int_{-1}^1 \frac{\omega^2 \mathrm{d} \omega\,\mathrm{d} \cos{\theta}}{(2\pi)^2}\, \cfrac{1}{2\omega} e^{- i \omega (t-t') + i \omega |\vec{x}-\vec{x}'| \cos{\theta}}\\
	&= \cfrac{1}{2 i |\vec{x}-\vec{x}'|}\int_0^\infty \frac{\mathrm{d} \omega}{(2\pi)^2}\, e^{- i \omega (t-t')} \left(e^{i\omega |\vec{x}-\vec{x}'|} - e^{-i\omega |\vec{x}-\vec{x}'|}\right)\\
\end{align}

This oscillating integral does not converge. Therefore we will first calculate \(D^+(x,x')\) for complex times by setting \(t \to t - i\varepsilon, \varepsilon > 0\) and then treating \(D^+(x,x')\) as a distribution when setting \(\varepsilon \to 0\).

\begin{align}
D^+(x,x') &= \cfrac{1}{2 i |\vec{x}-\vec{x}'|}\int_0^\infty \frac{\mathrm{d} \omega}{(2\pi)^2}\, e^{- i \omega (t-t' - i\varepsilon - |\vec{x}-\vec{x}'|)} - e^{- i \omega (t-t' - i\varepsilon + |\vec{x}-\vec{x}'|)}\\
	&= -\cfrac{1}{2 i |\vec{x}-\vec{x}'|}\frac{1}{(2\pi)^2}\left(\frac{i}{t-t' - i\varepsilon - |\vec{x}-\vec{x}'|} - \frac{i}{t-t' - i\varepsilon + |\vec{x}-\vec{x}'|}\right)\\
	&= -\cfrac{1}{2 |\vec{x}-\vec{x}'|}\frac{1}{(2\pi)^2} \frac{(t-t' - i\varepsilon + |\vec{x}-\vec{x}'|) - (t-t'-i\varepsilon - |\vec{x}-\vec{x}'|)}{(t-t'-i\varepsilon)^2 - |\vec{x}-\vec{x}'|^2}\\
	&= -\frac{1}{4\pi^2} \frac{1}{(t-t'-i\varepsilon)^2 - |\vec{x}-\vec{x}'|^2}
\end{align}

Alternatively one can achieve the form by first choosing a coordinate system such that \(\vb{x} = \tilde{t} \partial_0\) (as we will do for the thermal case) and then transforming it back (one has to be very careful with the \(\varepsilon\) here).  

\section{Thermal Wigthmanfunction in Minkowskispace}
To calculate the thermal Wightmanfunction we will restrict to the interior of the lightcone (we will only need that). At a specific point choose the following coordinate system: First we will set \(\vb{x}' = 0\) and then choose a coordinate system such that \(\vb{x} = \tilde{t} \partial_0\). Here \(\tilde{t} = \pm \sqrt{-\vb{x}^2}\) with the upper sign for \(t > 0\) and the lower for \(t < 0\). In this coordinate system we can redo the calculation of the Wightmanfunction and achieve
\begin{align}
D^+(\vb{x},0) = -\frac{1}{4\pi^2} \frac{1}{(\tilde{t} - i\varepsilon)^2}
\end{align} 

To find the thermal Wightmanfunction we to replace \(\tilde{t} \to \tilde{t} - i \beta n\)\todo{link} and add all the contributions (note that inside the lightcone \(D^+ = D^{(1)}\))
\begin{align}
D^+_\beta(\vb{x},0) &= -\frac{1}{4\pi^2} \sum_{n=-\infty}^\infty \frac{1}{(\tilde{t} -i\beta n - i\varepsilon)^2}\\
	&= \frac{1}{4\pi^2\beta^2} \sum_{n=-\infty}^\infty \frac{1}{(- i \frac{\tilde{t}}{\beta} - n - \varepsilon)^2}\\
	&= \frac{1}{4\beta^2} \frac{1}{\sin[2](- i \pi \frac{\tilde{t}}{\beta} - \varepsilon)}\\
	&= -\frac{1}{4\beta^2} \frac{1}{\sinh[2](\frac{\pi}{\beta}(\tilde{t} - i\varepsilon))}
\end{align}

Now go back to the old frame and replace \(\tilde{t} = \pm \sqrt{-\vb{x}^2}\)
\begin{align}
D^+_\beta(\vb{x},\vb{x}') &= -\frac{1}{4\beta^2} \frac{1}{\sinh[2](\frac{\pi}{\beta}\qty(\pm \sqrt{(t-t')^2 - |\va{x}-\va{x}'|^2} - i\varepsilon))}\\
	&= -\frac{1}{4\beta^2} \frac{1}{\sinh[2](\frac{\pi}{\beta}\sqrt{(t-t')^2 - |\va{x}-\va{x}'|^2})}
\end{align}

In the last step we set \(\varepsilon \to 0\) to get a closed form. One needs to be careful because the \(\varepsilon\) will move poles in different directions for \(t > 0\) and \(t < 0\). However normally we only have to worry about the pole at zero which will as in the vacuum case will be moved to the upper half\footnote{To show this recall that the pole at zero is due to the pole of the vacuum Wightmanfunction which contributes to sum as \(n = 0\)}. 

\section{The Unruh-Detector}
\todo{shorten?}
Our treatment of such a detector will follow (Quelle: Birell Davies).
One describes a detector by a operator \(m(\tau)\) which couples to the field via a interaction term \(c\cdot m(\tau) \phi(x(\tau))\), where \(c\) is small and \(x(\tau)\) is the trajectory of the detector. For \(\tau \to -\infty\) the detector is in the groundstate \(\ket{E_0}\) and the field is in the vacuum state \(\ket{0}\). The detector develops with time according to \(m(\tau) = e^{i H_0 \tau} m(0) e^{-i H_0 \tau}\) with \(H_0 \ket{E} = E\ket{E}\).\\
We would like to calculate the probability that the detector detects a particle with energy \(E\). Since \(c\) is small one can use first order perturbation theory where the transition amplitude to another state \(\ket{E,\psi}\) at time \(\tau\) is given by
\begin{align}
A_{\ket{E_0,0}\to\ket{E,\psi}}(\tau) &= i c \bra{E,\psi} \int_{-\infty}^\tau m(\tau') \phi(x(\tau'))\mathrm{d}\tau'\ket{E_0,0}\\
	&= i c \bra{E,\psi} \int_{-\infty}^\tau e^{i H_0 \tau'} m(0) e^{-i H_0 \tau'} \phi(x(\tau'))\mathrm{d}\tau'\ket{E_0,0}\\
	&= i c \bra{\psi} \int_{-\infty}^\tau e^{i E \tau'} \bra{E}m(0)\ket{E_0}  e^{-i E_0 \tau'} \phi(x(\tau'))\mathrm{d}\tau'\ket{0}\\
	&= i c \bra{E}m(0)\ket{E_0} \int_{-\infty}^\tau e^{i (E-E_0) \tau'} \bra{\psi}\phi(x(\tau'))\ket{0}\mathrm{d}\tau'\\
\end{align}

The transition probability is \(P_{\ket{E_0,0}\to\ket{E,\psi}}(\tau) = |A_{\ket{E_0,0}\to\ket{E,\psi}}(\tau)|^2\). But since we are only interested in the state of the detector we sum over all field configurations:
\begin{align}
P_E(\tau) &:= \sum_{i} P_{\ket{E_0,0}\to\ket{E,\psi_i}}(\tau) = \sum_{i}  |A_{\ket{E_0,0}\to\ket{E,\psi}}(\tau)|^2\\
		  &= c^2 |\bra{E}m(0)\ket{E_0}|^2 F_{E-E_0}(\tau)\\
\text{with}\,F_E(\tau) &= \sum_{i}\left|\int_{-\infty}^\tau e^{i (E-E_0) \tau} \bra{\psi_i}\phi(x(\tau'))\ket{0}\mathrm{d}\tau'\right|^2\\
	&= \sum_{i} \int_{-\infty}^\tau e^{-i E \tau''} \bra{0}\phi(x(\tau''))\mathrm{d}\tau''\ket{\psi_i}\bra{\psi_i}\int_{-\infty}^\tau e^{i E \tau'} \phi(x(\tau'))\ket{0}\mathrm{d}\tau'\\
	&= \int_{-\infty}^\tau\mathrm{d}\tau' \int_{-\infty}^\tau \mathrm{d}\tau'' e^{-i E (\tau''-\tau')} \bra{0}\phi(x(\tau'')) \phi(x(\tau'))\ket{0} := \int_{-\infty}^\tau\mathrm{d}\tau' \int_{-\infty}^\tau \mathrm{d}\tau'' e^{-i E (\tau''-\tau')} D^+(x(\tau''), x(\tau'))
\end{align}

There we introduced the Wightman function \(D^+(x,x') = \bra{0}\phi(x) \phi(x')\ket{0}\). The probability splits in a product of two parts. The first one only depends on the model of the detector while the second part only depends on the trajectory. We will therefore interpret the (so called detector response) function \(F_E(\tau)\) as the distribution of energy excitations as been 'seen' by an observer on the trajectory \(x(\tau)\).

The transition rate is then given by:
\begin{align}
\cfrac{\mathrm{d}F_E(\tau)}{\mathrm{d}\tau} &= \int_{-\infty}^\tau \mathrm{d}\tau'' e^{-i E (\tau''-\tau)} D^+(x(\tau''), x(\tau)) \phi(x(\tau))\ket{0} + \int_{-\infty}^\tau \mathrm{d}\tau' e^{-i E (\tau-\tau')} D^+(x(\tau), x(\tau'))\\
&= \int_{-\infty}^\tau \mathrm{d}\tau' e^{-i E (\tau'-\tau)} D^+(x(\tau'), x(\tau)) + e^{-i E (\tau-\tau')} D^+(x(\tau), x(\tau'))\\
&\overset{\tilde{\tau} = \tau'-\tau}{=} \int_{-\infty}^0 \mathrm{d}\tilde{\tau} e^{-i E \tilde{\tau}} D^+(x(\tilde{\tau} + \tau), x(\tau)) + e^{i E \tilde{\tau}} D^+(x(\tau), x(\tilde{\tau} + \tau))\\
&= 2 \mathrm{Re} \int_{-\infty}^0 \mathrm{d}\tilde{\tau} e^{-i E \tilde{\tau}} D^+(x(\tilde{\tau} + \tau), x(\tau))
\end{align}

since \(D^+(x,x')^* = D^+(x',x)\). For the special case that the Wightman function does only depend on the difference of the \(\tau\text{'s}\), i.e. \(D^+(x(\tau_1 + \tau'),x(\tau_2 + \tau')) = D^+(x(\tau_1),x(\tau_2))\) one can simplify this further:

\begin{align}
\cfrac{\mathrm{d}F_E(\tau)}{\mathrm{d}\tau} &=  \int_{-\infty}^0 \mathrm{d}\tilde{\tau} e^{-i E \tilde{\tau}} D^+(x(\tilde{\tau} + \tau), x(\tau)) + \int_{0}^\infty \mathrm{d}\tilde{\tau} e^{- i E \tilde{\tau}} D^+(x(\tau), x(\tau - \tilde{\tau}))\\
&= \int_{-\infty}^0 \mathrm{d}\tilde{\tau} e^{-i E \tilde{\tau}} D^+(x(\tilde{\tau} + \tau), x(\tau)) + \int_{0}^\infty \mathrm{d}\tilde{\tau} e^{- i E \tilde{\tau}} D^+(x(\tau  + \tilde{\tau}), x(\tau))\\
&= \int_{-\infty}^\infty \mathrm{d}\tilde{\tau} e^{-i E \tilde{\tau}} D^+(x(\tilde{\tau} + \tau), x(\tau)) = \int_{-\infty}^\infty \mathrm{d}\tilde{\tau} e^{-i E \tilde{\tau}} D^+(x(\tilde{\tau}), x(0))
\end{align}

The rate is the fouriertransform of the Wightman function and is independent of \(\tau\).

\section{Solving the geodesic equation in a two dimensional metric}
\label{sec:app_congruence}
Let us first assume we already solved the geodesic equation over the whole two dimensional spacetime (call it \(\mathcal{M}\)). Then we can construct lightcone coordinates: Fix a point \(\vb{x}_0\) in the spacetime. In this point there exist two linear independent null vector namely \(\vb{t}\) and another one defined by \(\vb{n}^2 = 0\) and \(\vb{t}\cdot\vb{n} = -1\). Note that since we are in a two dimensional spacetime \(\vb{n}\) is uniquely defined. Then solve for the geodesic starting at \(\vb{x}_0\) with tangent vector \(\vb{n}\). Associate a point on the geodesic with the corresponding value of the affine parameter \(\lambda\). Then starting at such a point solve the geodesic (call it \(\Sigma\)) with tangent vector \(\vb{t}\) and associate every point on this geodesic with the value of the affine parameter \(\tau\). By this we can find a map \(\vb{x}(\tau,\lambda)\). We can use this map for a coordinate transformation which yields to the coordinate system \(\partial_\tau = \vb{t}\) and \(\partial_\lambda = \vb{n}\).

For the boundary condition take \(\vb{t}\vb{n} = -1\) on \(\Sigma\). Before we start let us summarize the known properties of \(\vb{t}\) and \(\vb{n}\): Clearly on the whole spacetime \(\nabla_{\vb{t}}\vb{t} = 0\) and \(\vb{t}^2 = 0\). Also since they are a coordinate system \([\vb{t},\vb{n}] = 0\) which means \(\covd{t}\vb{n} = \nabla_{\vb{n}}\vb{t}\). On \(\Sigma\) we also know that \(\vb{t}\vb{n} = -1\) and \(\vb{n}^2 = 0\).

First observe that
\begin{align}
\vb{t} \covd{t}\vb{n} = \vb{t} \covd{n}\vb{t} = \frac{1}{2} \covd{n}\vb{t}^2 = 0
\label{equ:congruence_tdtn}
\end{align}
in the whole spacetime. Use this to calculate \( \covd{t}(\vb{t}\vb{n}) = \vb{t}\covd{t}\vb{n} = 0\) which means that
\begin{align}
\vb{t}\vb{n} = \mathrm{const.} = -1 
\end{align}
This also implies that \(\covd{t}\vb{n} = -\vb{t} (\vb{n}\covd{t}\vb{n})\,\)(since \(\vb{t} \covd{t}\vb{n} = 0\) and therefore \(\covd{t}\vb{n} \sim \vb{t}\)).

Next derive
\begin{align}
\vb{n}\covd{n}\vb{t} = \covd{n}(\vb{t}\vb{n} - \vb{t}\covd{n}\vb{n} = \covd{n}(-1) - \vb{t}\covd{n}\vb{n}) = - \vb{t}\covd{n}\vb{n}
\end{align}

Note that on \(\Sigma\): \(\covd{n}\vb{n} = 0\) which means \(\vb{n}\covd{n}\vb{t} = 0\) and (by eq. \ref{equ:congruence_tdtn} )\(\covd{n}\vb{t} = \covd{t}\vb{n} = 0\). Unfortunately this parallel transport condition is only satisfied on \(\Sigma\) not on \(\mathcal{M}\). Therefore two things will happen to \(\vb{n}\): it won't remain a null vector and it will not solve the geodesic equation outside of \(\Sigma\). To see this calculate
\begin{align}
\covd{t}\covd{t}\vb{n}^2 &= 2\covd{t}(\vb{n}\covd{t}\vb{n}) = 2(\covd{t}\vb{n})^2 + 2 \vb{n}\covd{t}\covd{t}\vb{n}\\
	&\overset{\covd{t}\vb{n} \sim \vb{t}}{=} 2 \vb{n}\covd{t}\covd{n}\vb{t} = 2\vb{n} R(\vb{t},\vb{n})\vb{t} + 2 \vb{n}\covd{n}\covd{t}\vb{t}\\
	&\overset{\covd{t}\vb{t} = 0}{=} 2 \vb{n}\mathrm{R}(\vb{t},\vb{n})\vb{t}
\end{align}

where \(\mathrm{R}(\vb{a},\vb{b}) = \covd{a}\covd{b}-\covd{b}\covd{a} - \nabla_{[\vb{a},\vb{b}]}\) is the curvature tensor. If it vanishes \(\dv[2]{\vb{n}^2}{\tau} = 0\) and so \(\vb{n}^2 = a\tau + b\). But from the boundary condition on \(\Sigma\) follows that \(a = b = 0\) and so \(\vb{n}^2 = 0\). However when there is curvature (as in our case) \(\vb{n}^2\) will differ from \(0\). Since the behaviour of \(\vb{n}^2\) fully determines \(\covd{n}\vb{n}\)
\begin{align}
\vb{t}\covd{n}\vb{n} &= \covd{n}(\vb{t}\vb{n}) - \vb{n}\covd{n} \vb{t} = - \vb{n}\covd{t}\vb{n} = -\frac{1}{2} \covd{t}\vb{n}^2\\
\vb{n}\covd{n}\vb{n} &= \frac{1}{2} \covd{n}\vb{n}^2
\end{align}  

this also means that \(\covd{n}\vb{n} \neq 0\). However is the curvature is small one can neglect this change and so \(\vb{n}^2 \approx 0\) and \(\covd{n}\vb{n} \approx 0\). One may then keep track of two neighbouring geodesics by computing the null geodesic between them and evaluating it at the corresponding \(\lambda\) value. Frankly speaking this means that the (null geodesic) distance \(\lambda\) between null geodesics will remain constant.

\section{Calculating the scalarproduct between early and later modes}
\label{sec:app_scalarproduct}
We would like to calculate the scalarproduct between the following modes in the later spacetime:

\begin{align}
u_{\omega l m} &= \frac{i^{-l}}{2i\sqrt{\pi\omega}r} e^{-i\omega u} Y_l^m (\theta, \phi) - \frac{i^{l}}{2i\sqrt{\pi\omega}r} e^{-i\omega v} Y_l^m (\theta, \phi)\\
\psi_{\omega l m} &= \frac{i^{-l}}{2i\sqrt{\pi\omega}r} e^{i\omega B e^{-\frac{u}{4M}}} Y_l^m (\theta, \phi) - \frac{i^{l}}{2i\sqrt{\pi\omega}r} e^{-i\omega v} Y_l^m (\theta, \phi)
\end{align}

To do so we choose as hypersurface a lightlike surface with \(v = \mathrm{const}.\) together with some spacelike surface that captures the complete interior region (we need both surfaces since \(\Im^+\) is only a partial Cauchysurface \footnote{Note that \(v = \mathrm{const}.\) is a lightlike surface but partial Cauchysurfaces need to be spacelike. So it is not a partial Cauchysurface. However as eq. \ref{equ:qft_scalarproduct_invariant} is also true for lightlike surfaces the value of the scalarproduct will not change.}). However the the \(u\) modes are \(0\) inside the black hole and therefore the integral over the interior will vanish.

So we will only integrate over the lightlike surface. The normal vector of the surface is given by \(S = \frac{r^2 \sin\theta}{f(r)} \partial_u\). We will neglect the factor \(f(r) \approx 1\) since the bigger part of the hypersurface will be far away from the black hole and we assume that the non approximate wave functions will drop to zero at the event horizon\footnote{Actually Hawking and all other authors I encountered so far didn't mentioned this factor.}.

Before evaluating the integral let us rewrite the modes. First define the prefactor as \(A = \frac{i^{-l}}{2i\sqrt{\pi\omega}}\) and then  
\begin{align}
u_{\omega l m} &= \frac{\tilde{u}_\omega}{r} Y_l^m (\theta, \phi)\\
\psi_{\omega l m} &= \frac{\tilde{\psi}_\omega}{r} Y_l^m (\theta, \phi)\\
\tilde{u}_{\omega} &= A e^{-i\omega u} + A^* e^{-i\omega v}\\
\tilde{\psi}_{\omega} &= A e^{i\omega B e^{-\frac{u}{4M}}} + A^* e^{-i\omega v}
\end{align}

Using this we can simplify the scalarproduct (we will drop the indices \(l, m\) because the angular integral will just give \(\delta_{ll'}\delta_{mm'}\)):
\begin{align}
(u_{\omega'}|\psi_\omega) &= i\int_{-\infty}^\infty r^2 \dd{u} \frac{\tilde{u}_{\omega'}^*}{r} \partial_u \frac{\tilde{\psi}_\omega}{r} - \frac{\tilde{\psi}_\omega}{r} \partial_u \frac{\tilde{u}_{\omega'}^*}{r}\\
	&= i\int_{-\infty}^\infty r^2 \dd{u} \frac{\tilde{u}_{\omega'}^*}{r} \frac{\partial_u \tilde{\psi}_\omega}{r} - \frac{\tilde{\psi}_\omega}{r} \frac{\partial_u \tilde{u}_{\omega'}^*}{r} - \frac{\tilde{u}_{\omega'}^*}{r} \frac{\tilde{\psi}_\omega}{r^2} \partial_u r + \frac{\tilde{\psi}_\omega}{r} \frac{\tilde{u}_{\omega'}^*}{r^2} \partial_u r\\
	&= i\int_{-\infty}^\infty \dd{u} \tilde{u}_{\omega'}^* \partial_u \tilde{\psi}_\omega - \tilde{\psi}_\omega \partial_u \tilde{u}_{\omega'}^*\\
	&= -2i\int_{-\infty}^\infty \dd{u} \tilde{\psi}_\omega \partial_u \tilde{u}_{\omega'}^*
\end{align}

In the last step we integrated by parts and assume that the boundary terms vanish (We know that the later modes drop to zero at the horizon \(u = \infty\). For \(u = -\infty\) we have a rapidly oscillating function which is zero at average).

\begin{align}
(u_{\omega'}|\psi_\omega) &= -2i\int_{-\infty}^\infty \dd{u} \tilde{\psi}_\omega \partial_u \tilde{u}_{\omega'}^*\\
	&= 2\omega' \int_{-\infty}^\infty \dd{u} \qty(A e^{i\omega B e^{-\frac{u}{4M}}} + A^* e^{-i\omega v}) A'^* e^{i\omega' u}\\
	&= 2\omega' AA'^* \int_{-\infty}^\infty \dd{u} e^{i\omega B e^{-\frac{u}{4M}}} e^{i\omega' u} + 2\omega' A^*A'^* e^{-i\omega v} \delta(\omega')\\
	&= 2\omega' AA'^* \int_{-\infty}^\infty \dd{u} e^{i\omega B e^{-\frac{u}{4M}}} e^{i\omega' u}
\end{align}

Next substitute \(x = e^{-\frac{u}{4M}}\) and then use contour integration to integrate over the positive imaginary axis (\(x = i y\)):
\begin{align}
(u_{\omega'}|\psi_\omega) &= 8M \omega' A^*A'^* \int_{0}^\infty \frac{\dd{x}}{x} e^{i\omega B x} e^{-4M i \omega' \ln x}\\
&= 8M \omega' AA'^* \int_{0}^\infty \dd{x} e^{i\omega B x} x^{-4M i \omega' - 1}\\
&= i 8M \omega' AA'^* \int_{0}^\infty \dd{y} e^{-\omega B y} (iy)^{-4M i \omega' - 1}\\
&= i^{-4M i \omega'} 8M \omega' AA'^* \int_{0}^\infty \dd{y} e^{-\omega B y} y^{-4M i \omega' - 1}\\
&\overset{z = \omega B y}{=} i^{-4M i \omega'} 8M \omega' AA'^* \int_{0}^\infty \frac{\dd{z}}{B\omega} e^{-z} z^{-4M i \omega' - 1} (\omega B)^{4M i \omega' + 1}\\
&= i^{-4M i \omega'} (\omega B)^{4M i \omega'} 8M \omega' AA'^* \Gamma(-4M i \omega')
\label{equ:app_scalarproduct}
\end{align}

To calculate \((u_{\omega'}|\psi_\omega*)\) one can redo the same calculation but choose a contour over the negative imaginary axis (\(x = - iy\))

\begin{align}
(u_{\omega'}|\psi_\omega^*) &= -2i\int_{-\infty}^\infty \dd{u} \tilde{\psi}_\omega^* \partial_u \tilde{u}_{\omega'}^*\\
	&= 2\omega' A^*A'^* \int_{-\infty}^\infty \dd{u} e^{-i\omega B e^{-\frac{u}{4M}}} e^{i\omega' u}\\
	&= 8M \omega'A^*A'^* \int_{0}^\infty \dd{x} e^{-i\omega B x} x^{-4M i \omega' - 1}\\
	&= -i 8M \omega' A^*A'^* \int_{0}^\infty \dd{y} e^{-\omega B y} (-iy)^{-4M i \omega' - 1}\\
	&= i^{4M i \omega'} 8M \omega' A^*A'^* \int_{0}^\infty \dd{y} e^{-\omega B y} y^{-4M i \omega' - 1}\\
	&\overset{z = \omega B y}{=} i^{4M i \omega'} 8M \omega' A^*A'^* \int_{0}^\infty \frac{\dd{z}}{\omega B} e^{-z} z^{-4M i \omega' - 1} (\omega B)^{4M i \omega' + 1}\\
	&= i^{4M i \omega'} (\omega B)^{4M i \omega'}  8M \omega' A^*A'^* \Gamma(-4M i \omega')\\
	&= (-1)^{l+1} i^{8M i \omega'} (u_{\omega'}|\psi_\omega) = (-1)^{l+1} e^{-4\pi M \omega'} (u_{\omega'}|\psi_\omega) 
\end{align}

So both scalar products lead (up to a prefactor) to the same result.

In eq. \ref{equ:hawking_bb} it was shown that \(\bra*{0}b_{\tilde{\omega}} b_{\omega'}\ket*{0}\) vanishes for \(\tilde{\omega} \neq \omega'\). For \(\tilde{\omega} = \omega'\) we still need to evaluate the integral directly:

\begin{align}
\bra*{0}b_{\tilde{\omega}} b_{\omega'}\ket*{0} &= (-1)^{l+1} e^{-4 M\pi \omega'} \int_0^\infty \dd{\omega} (u_{\omega'}|\psi_\omega) (u_{\omega'}|\psi_\omega)\\
&= (-1)^{l+1} e^{-4 M\pi \omega'} \int_0^\infty \dd{\omega} i^{-8M i \omega'} (\omega B)^{8M i \omega'} (8M)^2 \omega'^2 A^2A'^{2*} \Gamma(-4M i \omega')^2\\
	&= B^{8M i \omega'} (8M)^2 \omega'^2 \frac{1}{4\pi} A'^{2*} \Gamma(-4M i \omega')^2 \int_0^\infty \dd{\omega}\omega^{8M i \omega'} \frac{1}{\omega}\\
	&= (8M)^2 B^{8M i \omega'} \omega'^2 \frac{1}{4\pi} A'^{2*} \Gamma(-4M i \omega')^2 \int_0^\infty \dd{\omega}e^{(8M i \omega' - 1)\ln \omega}\\
	&\overset{x = \ln\omega}{=} B^{8M i \omega'} (8M)^2 \omega'^2 \frac{1}{4\pi} A'^{2*} \Gamma(-4M i \omega')^2 \int_{-\infty}^\infty \dd{x} e^x e^{(8M i \omega' - 1)x}\\
	&= B^{8M i \omega'} (8M)^2 \omega'^2 \frac{1}{4\pi} A'^{2*} \Gamma(-4M i \omega')^2 \delta(8M\omega')\\
	&= B^{8M i \omega'} 8M \omega'^2 \frac{1}{4\pi} A'^{2*} \Gamma(-4M i \omega')^2 \delta(\omega')
\end{align}

There is only a contribution for \(\omega' = 0\) which we excluded from our analysis.
