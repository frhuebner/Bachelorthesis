\chapter{The Schwarzschild solution}
\section{The metric}
In this thesis we will only use the Schwarzschild-metric to describe stars and black holes (which means that they have no charge and no angular momentum). The metric for an spherical symmetric object with mass \(M\) is given by
\begin{align}
\dd s^2 &= -f(r)\dd{t^2} + \frac{1}{f(r)}\dd{r^2} + r^2 \dd{\Omega} &\dd{\Omega} = \dd{\theta^2} + \sin[2](\theta) \dd{\phi^2} 
\end{align}
where \(f(r) = 1-\frac{2M}{r}\). The metric is only valid outside the boundary of the star or for \(r > 2M\). The two vector fields \(\partial_t\) and \(\partial_\phi\) are killing.

\section{Lightlike radial geodesics}
We will later need lightlike radial geodesics in the Schwarzschild metric. These are defined by \(\dot{x}^2 = 0\), \(\dot{\theta} = \dot{\phi} = 0\), where the dot means derivative w.r.t the affine parameter \(\lambda\). Because \(\partial_t\) is killing we know that \(E := f(r)\dot{t}\) is constant along the geodesic. This yields to 
\begin{align}
0 &= -f(r)\dot{t}^2 + \frac{1}{f(r)}\dot{r}^2\\
  &= -\frac{1}{f(r)} (E^2 - \dot{r}^2)
\end{align}
Since \(f(r) \neq 0\) we conclude: 
\begin{align}
\dot{r} &= \pm E\\
r &= \pm E \lambda + r_0
\label{bh_geo_r_E}
\end{align}
Inserting this into the definition of \(E\) yields to
\begin{align}
\dot{r} &= \pm f(r) \dot{t}\\
0 &= \dot{t}\mp\frac{\dot{r}}{f(r)} = \dv{\lambda}(t \mp r_*)
\end{align}

where the tortoise coordinate \(r_* = r + 2M\ln(\frac{r}{2M} - 1)\) has the property \(\dv{r_*}{r} = \frac{1}{f(r)}\). So either \(u := t - r_*\) (outgoing geodesic) or \(v := t + r_*\) (ingoing geodesic) is constant.

Consider an ingoing null geodesic \(v = \mathrm{const}, r = -E\lambda + r_0\) (from \(\dot{r} = -E\)) and compute

\begin{align}
\dv{u}{\lambda} &=  \dv{2 t - v} = 2\dot{t} = 2\frac{E}{f(r)}\\
	&= 2\frac{Er}{r-2M} = 2E\frac{r-2M + 2M}{r-2M} = 2E\qty(1 + \frac{2M}{r-2M})\\
	&= 2E\qty(1 + \frac{2M}{r_0-2M - E\lambda})\\
u(\lambda) &= 2E\lambda - 4M\ln(r_0-2M-E\lambda) + 4M\ln(r_0-2M) + u_0\\
	&= 2E\lambda - 4M\ln(1 - \frac{E\lambda}{r_0-2M}) + u_0
\label{equ:bh_geo_in}
\end{align}

where \(u_0 = u(0)\). For an outgoing geodesic one finds analogously:
\begin{align}
v(\lambda) &= 2E\lambda + 4M\ln(1 + \frac{E\lambda}{r_0 - 2M}) + v_0
\label{equ:bh_geo_out}
\end{align}

We will later need the affine separation between null geodesics, i.e. the value \(\lambda\) on a lightlike geodesic running from one geodesic to the other. For two ingoing geodesics (characterised by \(v, v'\)) far away from the black hole (i.e. \(E\lambda \ll r_0 - 2M\)) one can neglect the logarithm and the distance from \(v_0\) to \(v\) is given by:
\begin{align}
\lambda = \frac{v - v_0}{2E} 
\label{equ:bh_in_simplified}
\end{align}

The other case which will be important is the distance of two outgoing geodesics \(u_0, u\) where \(u_0 = \infty\) is the event horizon \(r_0 \to 2M\). Now one can neglect the linear term and the distance (note that \(\lambda <0\)) is given by:
\begin{align}
u &= - 4M\ln(1 - \frac{E\lambda}{r_0-2M}) + u_0\\
	&= - 4M\ln(1 - \frac{E\lambda}{r_0-2M}) + v_0 - 2r_{0*}\\
	&= - 4M\ln(1 - \frac{E\lambda}{r_0-2M}) + v_0 - 2r_0 - 4M\ln(\frac{r_0-2M}{2M})\\
	&= - 4M\ln(\qty(1 - \frac{E\lambda}{r_0-2M})\frac{r_0-2M}{2M}) + v_0 - 2r_0\\
	&= - 4M\ln(\frac{r_0 - 2M - E\lambda}{2M}) + v_0 - 2r_0\\
	&= - 4M\ln(-\frac{E\lambda}{2M}) + v_0 - 4M\\
-\lambda &= \frac{2M}{E}e^{1+\frac{v_0}{4M}} e^{-\frac{u}{4M}}
\label{equ:bh_out_simplified}
\end{align} 

\section{The Klein-Gordon-Equation in the Schwarzschild metric}

Consider a static spherical star where the outer metric is the Schwarzschild metric (i.e. non rotating and uncharged). It's important that we are considering a star because a black hole does not provide a global timelike killing vector field (analytic extension of \(\partial_t\) leads to a spacelike vectorfield inside the black hole). From now on we will only consider the outer region. To achieve a global solution one need to match inner with outer solutions.

The Klein-Gordon-Equation \(\nabla_\mu\nabla^\mu \phi = 0\) can be written as:
\begin{align}
-\frac{r^2}{f(r)}\partial_t^2 \phi + \qty(\partial_r r^2 f(r) \partial_r) \phi - L^2 \phi = 0
\end{align}

where \(L^2\) is the usual angular momentum operator.

\subsection{Spherical Modes}

Since the spacetime is spherical symmetric and has the killing vector field \(\partial_t\) we can do the following ansatz for the modes
\begin{align}
u_{\omega l m} = A e^{-i\omega t} \frac{R_{\omega l}}{r}Y_l^m (\theta, \phi)
\end{align}

Plugging this into the Klein-Gordon-Equation \(\nabla_\mu\nabla^\mu u_{\omega l m} = 0\) yields to the following equation for \(R_{\omega l}\) (see for example (Birell Davies)):
\begin{align}
\dv[2]{R_{\omega l}}{r_*} + \omega^2 R_{\omega l} -\qty(\frac{l(l+1)}{r^2} + \frac{f'(r)}{r})f(r) R_{\omega l} &= 0\\
\dv[2]{R_{\omega l}}{r_*} + \omega^2 R_{\omega l} - \order{r^{-2}} R_{\omega l} &= 0
\end{align}

So for \(\omega r \gg l\) one can neglect the \(r\) dependent part. In this case we find the asymptotic solutions
\begin{align}
R_{\omega l} &= e^{\pm i \omega r_*}\\
u_{\omega l m} &=  \frac{A_{\omega l m}}{r} e^{-i\omega t + i\omega r_*} Y_l^m (\theta, \phi) + \frac{B_{\omega l m}}{r} e^{-i\omega t - i\omega r_*} Y_l^m (\theta, \phi)\\
	&= \frac{A_{\omega l m}}{r} e^{-i\omega u} Y_l^m (\theta, \phi) + \frac{B_{\omega l m}}{r} e^{-i\omega v} Y_l^m (\theta, \phi)
\end{align}

Unfortunately we either cannot determine the (quite important) phase between \(A_{\omega l m}\) and \(B_{\omega l m}\) nor can we normalise the modes by integrating over all space. Instead I will impose that very far away from the star (where \(r_* \approx r\)) the field behaves as in Minkowskispace (which means that e.g. all experiments give the same results). This implies that \(D^+(\vb{x},\vb{x}')\) is the same as in Minkowskispace. Comparing with the asymptotic spherical modes in Minkowskispace (see eq. \ref{equ:bessel_asympt}) yields to:
\begin{align}
u_{\omega l m} &=  \frac{1}{\sqrt{\pi\omega}} e^{-i\omega t} \frac{\sin(\omega r_* - l\frac{\pi}{2})}{r} Y_l^m (\theta, \phi)  = \frac{r_*}{r} u_{\omega l m}\upd{M}(r_*,\theta,\phi)\\
&= \frac{i^{-l}}{2i\sqrt{\pi\omega}r} e^{-i\omega u} Y_l^m (\theta, \phi) - \frac{i^{l}}{2i\sqrt{\pi\omega}r} e^{-i\omega v} Y_l^m (\theta, \phi)
\label{equ:bh_modes}
\end{align}

\subsection{The Wightman Function}
The next step would be to calculate the Wightman function \(D^+(\vb{x}, \vb{x}')\). Note that since \(\partial_t\) is a timelike killing vector we define the ground state \(\ket{0}\) by \(a_{\omega l m} \ket{0} = 0\) and use eq. \ref{equ:wightman_modes}.
\begin{align}
D^+(\vb{x}, \vb{x}') = \int_0^\infty \frac{\dd{\omega}}{\pi\omega} \sum_{l,m} e^{-i\omega(t-t')} \frac{\sin(\omega r_* - l\frac{\pi}{2})}{r} \frac{\sin(\omega r_*' - l\frac{\pi}{2})}{r'} Y_l^m(\theta,\phi) Y_l^{m*}(\theta',\phi')
\end{align}

Since this integral is nearly the same as in Minkowskispace (see eq. \ref{equ:wightman_minkoski_spherical}) we also encounter the same problems, namely the IR divergence and the fact that it is non-zero only in two directions. This is due to the fact that the approximation \(\omega r \gg l\) breaks down for small \(\omega\) and for large \(l\). Recall that the problems in the non approximate calculation in Minkowskispace didn't occur because the \(j_l(\omega r)\) remain finite at \(\omega \to 0\). In other words the essential feature of the \(j_l(\omega r)\) is that they let all the \(\cos(\omega ...)\) terms in the integral drop to zero for \(\omega \to 0\) instead of \(\cos{0} = 1\) in the approximate case. Therefore we can assume that the same happens for exact solutions around the star.

Unfortunately the exact behaviour for small \(\omega\) is the same as for small \(r\) which will depend on the specific geometry of the star. But the metric of a star is almost flat (since the radius of a star \(R_0\) is much bigger than its Schwarzschildradius \(R_S\)). Therefore we will approximate the global mode by replacing the sine with the spherical bessel function:
\begin{align}
u_{\omega l m} &= \frac{\sqrt{\omega}}{\sqrt{\pi}} e^{-i\omega t} \frac{r_*}{r} j_l(\omega r_*) Y_l^m (\theta, \phi) = \frac{r_*}{r} u_{\omega l m}\upd{M}(r_*,\theta,\phi)
\end{align}

Because the modes now are the same (up to a prefactor and replacing \(r \to r_*\)) as in Minkowskispace we can find the Wightman function by adjusting the Wightman function of Minkowskispace\todo{link}:
\begin{align}
D^+(\vb{x}, \vb{x}') &= -\frac{1}{4\pi^2}\frac{r_* r_*'}{r r'} \frac{1}{(t-t'-i\varepsilon)^2 - |\va{x}_*-\va{x}_*'|^2}\\
	&=  -\frac{1}{4\pi^2}\frac{r_* r_*'}{r r'} \frac{1}{(t-t'-i\varepsilon)^2 - r_*^2 - r_*'^2 + 2r_*r_*' \cos{\alpha}}
\end{align}

where \(\va{x}_*\) is obtained by replacing \(r \to r_*\) in \(\va{x}\) and \(\alpha\) is the angle between the two vectors. So the Wightmanfunction on a trajectory outside the star can be approximated by the Wightmanfunction in Minkowskispace on a slightly different trajectory.

\section{Observers in the Schwarzschildmetric}

\section{Static observer}
We already know by lemma \ref{lemma:static_spacetime} that a static observer will not see any excitations. However we can proof this explicitly in the Schwarzschildmetric. A static observer is given by \(t = \frac{\tau}{\sqrt{f(r)}}\) and all other coordinates constant. Inserting this into the Wightmanfunction yields
\begin{align}
D^+(\vb{x}(\tau), \vb{x}(\tau')) =  -\frac{1}{4\pi^2}\frac{r_*^2}{r^2} \frac{f(r)}{(\tau-\tau'-i\varepsilon)^2}
\end{align} 

Up to a prefactor this is exactly the same as in Minkowskispace \todo{link}. So a static observer indeed does not recognize any particles.
 
\section{Circular observer}
By Lemma \ref{lemma:static_spacetime} we know that a static observer (but nevertheless proper accelerating) won't see any excitations. But we also would like to know if a freely falling observer (unlike in Minkowskispace) will recognize particles.

A circular observer is given by \(t = a\tau\), \(\phi = B\tau\), \(r = \mathrm{const.}\), \(\theta = \frac{\pi}{2}\)\footnote{Since the spacetime is spherically symmetric we can choose a coordinate system such that \(\theta = \frac{\pi}{2}\) and \(B>0\)}. The geodesic equation and \(\dot{\vb{x}}^2 = 0\) give constrains on the constants
\begin{align}
A^2 &= \frac{r}{r-3M}\\
B^2 &= \frac{1}{r^2}\frac{M}{r-3M}
\end{align}

The Wightman function evaluated on the curve is 
\begin{align}
D^+(\vb{x}(\tau), \vb{x}(\tau')) &= -\frac{1}{4\pi^2}\frac{r_*^2}{r^2} \frac{1}{(A(\tau-\tau')-i\varepsilon)^2 - 2r_*^2 (1 - \cos{B(\tau-\tau')})}
\end{align}

We see that \(D^+\) only depends on \(\tau - \tau'\). So we can use the simplified formular \eqref{equ:detector_final}. For this we need to find the poles in the lower half of \(D^+(\vb{x})(\tau),0\) which means finding the roots of 

\begin{align}
0 &= A^2\tau^2 - 2r_*^2 (1 - \cos{B\tau})\\
0 &= \xi^2 x^2 - 2(1 - \cos{x})
\end{align}

where \(x = B\tau\) and \(\xi = \frac{A}{Br_*}\)\footnote{We exclude the case that \(r_* = 0\) since this corresponds to \(r\approx 1.4 \cdot 2M\) which is not far away from the horizon}. 
Clearly \(\tau = 0\) is a root. But we know that due to the \(\varepsilon\) (\todo{link}) this root will always lie in the upper half. Apart from that we have to differentiate between two cases\todo{more information}:
\begin{itemize}
\item \(\xi < 1\): In this case there are another two roots on the real axis. The reason is that for \(\xi < 1\) (which is a very fast circular motion) the trajectory hits the Minkowski-lightcone. But we know that this is not possible in the Schwarzschild-spacetime \todo{really?,link}. So we assume that this behaviour is due to our approximation and therefore exclude this case.
\item \(\xi > 1\): This case represents slower motions. There are two more (first order) poles at \(\pm i x_0\) of whom one is in the lower half. 
\end{itemize}

The rate is given by eq. \ref{equ:detector_final}
\begin{align}
\dv{F_E}{\tau} &= \int_{-\infty}^\infty \dd{\tau} e^{-i E \tau} D^+(\vb{x}(\tau), \vb{x}(0))\\
	&= -\frac{1}{4\pi^2}\frac{r_*^2}{r^2} \int_{-\infty}^\infty \dd{\tau} e^{-i E \tau} \frac{1}{A^2\tau^2 - 2r_*^2 (1 - \cos{B(\tau)})}\\
	&= -\frac{1}{4\pi^2}\frac{1}{r r' B} \int_{-\infty}^\infty \dd{x} e^{-i E x} \frac{1}{\xi^2 x^2 - 2 (1 - \cos{x})}\\
	&= \frac{i}{2\pi}\frac{1}{r r' B} \mathrm{Res}\qty(e^{-i E x} \frac{1}{\xi^2 x^2 - 2 (1 - \cos{x})}, -ix_0)\\
	&= \frac{i}{2\pi}\frac{1}{r r' B} e^{-E x_0} \lim_{x\to -i x_0} \frac{x + ix_0}{\xi^2 x^2 - 2 (1 - \cos{x})}\\
	&= \frac{1}{2\pi}\frac{1}{r r' B} e^{-E x_0} \frac{1}{-2\xi^2 x_0 + 2 \sinh{x_0}}
\end{align}

So basically a circular observer sees a exponentially falling energy distribution.  

    