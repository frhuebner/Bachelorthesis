\chapter{Introduction}
Since Hawking \cite{hawking} showed 1975 that during a collapse of a star to a black hole particle production occurs this topic is of great interest in theoretical physics. This effect is nowadays called ``Hawking Radiation'' and leads independent of the details of the collapse (at least up to the approximations used by Hawking) to a thermal state with the Hawking temperature \(T\ind{H} = \frac{1}{8\pi k\ind{B} M}\). The reason why ``Hawking Radiation'' is still interesting is that it could lead to complete evaporation of the black hole which implies a violation of unitary (the so called black hole information paradox) and QFT in curved spacetime is therefore believed to be self inconsistent.\cite{hawking}\cite{Townsend}

In this thesis we will calculate how an observer moving in the spacetime after the collapse encounters Hawking radiation. To do so we will use the Unruh detector model. In this model the measured spectrum is interpreted as particle excitations observed by the observer. As ``Hawking Radiation'' the measured spectrum is a global effect and will therefore not depend on the current state but rather on the whole wordline of the observer (In other words: We cannot apply the equivalence principle). \cite{davies}

For observers in the later spacetime one is mainly interested in their observed temperature (i.e. do they see ``Hawking Radiation'' with a different temperature). There have been some approaches yet. For example \cite{smerlak} used an adiabatic expansion to justify a local treatment (they also only considered s-wave contribution which means an effectively two dimensional field.). In \cite{Hodgkinson} the field equations were solved numerically to obtain the temperature (for static and circular observers). We will however choose a different approach namely using an asymptotic form of the field. Therefore all results will only be valid far away from the black hole but we neither need to rely on adiabaticity nor we need to solve the field equations numerically.  

The thesis is split in three parts. We will start with a short introduction on quantisation of scalar fields in curved spacetimes and some basic examples of the Unruh detector in Minkowski space. The second part will then consider general properties of the Unruh detector spectrum in a static spacetime. In this chapter we will also find concrete examples proving that the equivalence principle is not applicable here. In the last part we will use the general properties to analyse the observed spectrum for static, circular and radial observers before and after a collapse to a black hole. To determine the apparent temperature for such observers in the later spacetime we will develop sufficiently method and evaluate it numerically.