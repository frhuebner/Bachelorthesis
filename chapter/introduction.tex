\chapter{Introduction}
Since Hawking \cite{hawking} showed 1975 that during a collapse of a star to a black hole particle production occurs this topic is of great interest in theoretical physics. This effect is nowadays called ``Hawking Radiation'' and leads independent of the details of the collapse (at least up to the approximations used by Hawking) to a thermal state with the Hawking temperature \(T\ind{H} = \frac{1}{8\pi k\ind{B} M}\). The reason why ``Hawking Radiation'' is still interesting is that it could lead to complete evaporation of the black hole which implies a violation of unitary (the so called black hole information paradox) and QFT in curved spacetime is therefore believed to be self inconsistent.\cite{hawking}\cite{Townsend}

In this thesis we will calculate how an observer moving in the spacetime after the collapse encounters Hawking radiation. To do so we will use the Unruh detector model. In this model the measured spectrum is interpreted as particle excitations observed by the observer. As ``Hawking Radiation'' the measured spectrum is a global effect and will therefore not depend on the current state but rather on the whole wordline of the observer (In other words: We cannot apply the equivalence principle). \cite{davies}

The thesis is split in three parts. We will start with a short introduction on quantisation of scalar fields in curved spacetimes and some basic examples of the Unruh detector in Minkowski space. The second part will then consider general properties of the Unruh detector spectrum in a static spacetime. In this chapter we will also find concrete examples proving that the equivalence principle is not applicable here. In the last part we will use the general properties to analyse the observed spectrum for static, circular and radial observers before and after a collapse to a black hole. For observers in the later spacetime we are mainly interested in their observed temperature (i.e. do they see ``Hawking Radiation'' with a different temperature). Since we won't be able to find this analytically we will develop and use a numerical method to determine the temperature.  

We are left with comparing to results of other authors. There are only few calculations of the appearing temperature in the later spacetime using an Unruh detector. In \cite{deeg} the Unruh detector is treated in a local manner (justified by an adiabaticity condition) and the four dimensional spacetime is replaced by a two dimensional. While for static observers the result agrees the temperature shift differs by orders of magnitude for infalling observers. However one has to be quite careful using both alternations\todo{was heißt das?} above. In \cite{smerlak} the temperature was obtained by using an adiabatic expansion of the Wightmanfunction which was effectively two dimensional because only s-wave modes where used. They only gave the result for a circular observer which slightly differs from our but has the same order of magnitude as the expected error through the approximation. The detector response function has been evaluated numerically on static and circular trajectories in \cite{Hodgkinson}. They concluded in both cases that the temperature is given by the Tolman relation which is consistent with our result.  