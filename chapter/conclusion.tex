\chapter{Conclusion}
We showed properties of the Wightman function and the spectrum of an Unruh detector in a static spacetime. Using normal coordinates we were able to find that the singularity at \(\vb{x} = \vb{x}'\) will -- when evaluated on a curve -- give a second order pole \(-\frac{1}{4\pi^2 \tau^2}\) which is shifted to the upper half by a small regularisation \(\varepsilon\). This term does not contribute to the spectrum. We argued that apart from this singularity the Wightmanfunction on any timelike curve will remain finite.

Furthermore we found that in general in a static spacetime in the vacuum state a static observer does not encounter any particles. If the spacetime is in a thermal state with temperature \(T_0\) the observer sees a thermal spectrum with a different temperature according to the Tolmanrelation: \(T = \frac{T_0}{\sqrt{g_{tt}}}\). For an observer moving along a spatial killing vectorfield we found a condition when he will encounter particles in the vacuum state which is independent of whether the trajectory is a geodesic or not. This means that the particle spectrum is a global effect and cannot be treated in a local manner.    

We were able to apply this knowledge\todo{better} to find an approximate form of the Wightman function in the outer Schwarzschild metric. This approximation is valid for \(r > 200 M\). Using it we showed the general results from before explicitly, namely a static observer does not encounter any particles and a circular geodesic observer sees a non vanishing spectrum. Then we applied the result of Hawking that after the collapse to a black hole the field is in a thermal state with \(\beta = 8\pi M\). A method to determine the temperature measured by different observers moving in this spacetime was developed and applied to static, circular and radial infalling observers. In all cases the temperature followed the Tolman relation. For static and circular observers there was no difference up to numerical errors. For radial observers we found a slight deviation outside the numerical errors. However this deviation has the same order of magnitude as expected by our approximation. Therefore a better approximation would be necessary to decide whether this is significant or not. For the time being we can only conclude that for all considered observers the temperature is given by the Tolman relation.    

       

  