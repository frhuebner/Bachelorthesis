\chapter{Conclusion}
We showed properties of the Wightman function and the spectrum of an Unruh detector in a static spacetime. Using normal coordinates we were able to find that the singularity at \(\vb{x} = \vb{x}'\) will -- when evaluated on a curve -- give a second order pole \(-\frac{1}{4\pi^2 \tau^2}\) which is shifted to the upper half by a small regularisation \(\varepsilon\). This term does not contribute to the spectrum. We argued that apart from this singularity the Wightman function on any timelike curve will remain finite.

Moreover we found that in general in a static spacetime in the vacuum state a static observer does not encounter any particles. If the spacetime is in a thermal state with temperature \(T_0\) the observer sees a thermal spectrum with a different temperature according to the Tolman relation: \(T = \frac{T_0}{\sqrt{g_{tt}}}\). For an observer moving along a spatial killing vectorfield we found a condition when particles are encountered in the vacuum state which is independent of whether the trajectory is a geodesic or not. In particular, this means the particle spectrum is a global effect and cannot be treated in a local manner.    

We were able to apply these results to find an approximate form of the Wightman function in the outer Schwarzschild metric. This approximation is valid for \(r \gtrsim 200 M\). Using it we showed the general results from before explicitly: a static observer does not encounter any particles and a circular geodesic observer sees a non vanishing spectrum. Then we applied Hawking's result which states that after the collapse to a black hole the field is in a thermal state with \(\beta = 8\pi M\). A method to determine the temperature measured by different observers moving in this spacetime was developed and applied to static, circular and radial infalling observers. In all cases the temperature followed the Tolman relation. For static and circular observers there was no difference up to numerical errors. For radial observers we found a slight deviation outside the numerical errors. However this deviation has the same order of magnitude as expected by our approximation. Therefore a better approximation would be necessary to decide whether this is significant or not. For the time being we can only conclude that for all considered observers the temperature is given by the Tolman relation.    

We conclude our discussion by comparing the given results to those mentioned in the introduction. Both approaches also yield that the Tolman relation describes the temperature for static observers. In \cite{smerlak} (local treatment of the detector) a result is only stated explicitly for a circular observer which slightly differs from ours. But the difference has the same order of magnitude as the expected error through our approximation. So again here one would need a more exact Wightman function to find whether both results agree. The numerical approach in \cite{Hodgkinson} -- which is more similar to our method since it is also non local -- leads to an observed temperature given by the Tolman relation for circular observers which is consistent with our result.    

  