\chapter{Collapsing Black Holes -- The Hawking Effect}
\missingfigure{}

In this chapter we will calculate the transition rate for an detector moving in the spacetime after a star collapsed into a black hole. We will first follow the treatment of Hawking (Quelle: Hawking) to show that the quantum field after the collapse is equivalent to a thermal state in the former spacetime. Using this we then calculate the transition rate on various trajectories. 

\section{Hawking radiation seen by observers}

In section \todo{link} we found the thermal Wightmanfunction \(D^+_\beta\) for the Minkowskispace. Again we can find the Wightmanfunction for the black hole by replacing \(r \to r_*\):
\begin{align}
D^+_\beta(\vb{x},\vb{x}') &= -\frac{1}{4\beta^2} \frac{r_*r_*'}{r r'} \frac{1}{\sinh[2](\frac{\pi}{\beta}\sqrt{(t-t')^2 - |\va{x}_*-\va{x}_*'|^2})}
\end{align}

\subsection{Static observer}
For a static observer the Wightman function is given by\todo{link}\todo{redo with thermal greens function}:
\begin{align}
D^+(\vb{x}(\tau), \vb{x}(\tau')) =  -\frac{1}{4\pi^2}\frac{r_*^2}{r^2} \frac{1}{\qty(\frac{\tau-\tau'}{\sqrt{f(r)}}-i\varepsilon)^2}
\end{align}

We then need to replace \(t = \frac{\tau}{\sqrt{f(r)}} \to t - i\beta n = \frac{\tau}{\sqrt{f(r)}} - i\beta n\):
\begin{align}
-\frac{1}{4\pi^2}\frac{r_*^2}{r^2} \frac{f(r)}{\qty(\tau - i\beta\sqrt{f(r)} n - i\varepsilon)^2}
\end{align}

This is up to a prefactor the same as for a static observer in Minkowskispace \todo{link} so the result will be analogously
\begin{align}
\frac{1}{2\pi} \frac{r_*^2 f(r)}{r^2} \frac{E}{e^{\beta\sqrt{f(r)} E}-1}
\end{align}

\todo{Amplitude}

The most important difference to Minkowskispace is that the measured temperature changed according to \(T = f(r)^{-1/2} T\ind{H}\) due to the fact that the proper time differs from the coordinate time. This happens in general in a heat bath on a spacetime and is called the Tolman effect\todo{Quelle}.  

\subsection{Circular orbiting observer}
An observer on a circular orbit \(t = A \tau\) and \(\phi = B\tau\) has the following thermal Wightman function:
\begin{align}
D^+_\beta(\vb{x}(\tau),\vb{x}(0)) &= -\frac{1}{4\beta^2} \frac{r_*^2}{r^2} \frac{1}{\sinh[2](\frac{\pi}{\beta}\sqrt{A^2\tau^2 - 2 r_*^2 (1-\cos{B\tau})})}\\
&= -\frac{1}{4\beta^2} \frac{r_*^2}{r^2} \frac{1}{\sinh[2](\frac{\pi}{\beta}r_*\sqrt{\xi^2 x^2 - 2 (1-\cos{x})})}
\end{align}

where we replaced \(x = B\tau\) and \(\xi = \frac{A}{Br_*}\) as in \todo{link}.

Because of the square root the function is not analytical and we can't use the residue theorem as before. We need to integrate the function by hand. Before that we need to get rid of the singularity at \(x = 0\). This is a second order pole so the series expansion will look like \(\frac{a_{-2}}{x^2} + \order*{x^0}\). \(a_{-2}\) can be computed by taking the limit
\begin{align}
a_{-2} &= \lim_{x\to 0}\frac{x^2}{\sinh[2](\frac{\pi}{\beta}r_*\sqrt{\xi^2 x^2 - 2 (1-\cos{x})})}\\
	&= \lim_{x\to 0}\frac{x^2}{\qty(\frac{\pi}{\beta}r_*\sqrt{\xi^2 x^2 - 2 (1-\cos{x})})^2}\\
	&= \qty(\frac{\beta}{\pi r_*})^2 \lim_{x\to 0}\frac{x^2}{\xi^2 x^2 - x^2 + \order*{x^4}}\\
	&= \qty(\frac{\beta}{\pi r_*})^2 \frac{1}{\xi^2 -1}
\end{align} 

The same value of \(a_{-2}\) is achieved when using \(\frac{1}{\sinh[2](\frac{\pi}{\beta}r_* \sqrt{\xi^2-1} x)}\). So we can write
\begin{align}
\frac{1}{\sinh[2](\frac{\pi}{\beta}r_*\sqrt{\xi^2 x^2 - 2 (1-\cos{x})})} &= \frac{1}{\sinh[2](\frac{\pi}{\beta}r_* \sqrt{\xi^2-1} x)} + g(x)
\end{align}
where 
\begin{align}
g(x) &= \frac{1}{\sinh[2](\frac{\pi}{\beta}r_*\sqrt{\xi^2 x^2 - 2 (1-\cos{x})})} - \frac{1}{\sinh[2](\frac{\pi}{\beta}r_* \sqrt{\xi^2-1} x)}
\end{align}

will remain finite at \(x = 0\) and will give some energy distribution on the detector. The other term however has the same form as before, so it will lead to a thermal excitation rate
\begin{align}
\frac{1}{2\pi} \frac{r_*^2}{r^2} \qty(A^2-B^2 r_*^2) \frac{E}{e^{\beta\sqrt{A^2-B^2 r_*^2} E}-1}
\end{align} 

Again the amplitude and the observed temperature changed \(T = \sqrt{A^2-B^2 r_*^2}^{-1} T_H\). 

.If one inserts the values for geodesics \todo{link} into \(\xi\) one finds
\begin{align}
\xi^2 = \frac{r^2}{r_*^2} \frac{r}{M} \approx  \frac{r}{M}
\end{align}
The last approximation is the result in newtons theory\footnote{In the newtonian limit \(\frac{1}{\xi}\) is the velocity of the observer which is usually very small compared to the velocity of light.}. So when we are some distance away from the black hole (like \(r \geq 100 M\)) we can assume that \(\frac{1}{\xi^2}\) is quite small. Therefore we will do a perturbative expansion in this parameter before the cosine term\todo{check}.   

\begin{align}
D^+_\beta(\vb{x}(\tau),\vb{x}(0)) &= -\frac{1}{4\beta^2} \frac{r_*^2}{r^2} \frac{1}{\sinh[2](\frac{\pi}{\beta}r_*\xi\sqrt{x^2 - \frac{2}{\xi^2} (1-\cos{x})})}\\
&= -\frac{1}{4\beta^2} \frac{r_*^2}{r^2} \frac{1}{\sinh[2](\frac{\pi}{\beta}r_*\xi x)} - \frac{1}{4\beta^2}\frac{r_*^2}{r^2} \frac{2\cosh(\frac{\pi}{\beta}r_*\xi x)}{\sinh[3](\frac{\pi}{\beta}r_*\xi x)} \frac{\pi r_* \xi}{\beta x} (1- \cos{x}) \cdot \frac{1}{\xi^2}\\
&= -\frac{1}{4\beta^2} \frac{r_*^2}{r^2} \frac{1}{\sinh[2](\frac{\pi}{\beta}r_*\xi x)} - \frac{1}{4\beta^2\xi}\frac{r_*^2}{r^2} \frac{2\cosh(\frac{\pi}{\beta}r_*\xi x)}{\sinh[3](\frac{\pi}{\beta}r_*\xi x)} \frac{\pi r_*}{\beta x} (1- \cos{x})
\end{align}
