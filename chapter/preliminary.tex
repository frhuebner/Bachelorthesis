\chapter{Preliminaries}
\section{Spacetimes}

\section{Klein-Gordon-Field}
(Quelle: Birell)
Consider a massless complex Klein-Gordon-field in a curved spacetime with metric \(g_{\mu\nu}\) given by the lagrangian:
\begin{align*}
\mathcal{L} &= \sqrt{|g|} g^{\mu\nu} \partial_\mu \phi\,\partial_\nu \phi 
\end{align*}

The equation of motion is given by the Euler-Lagrange-equation:
\begin{align*}
\sqrt{|g|}\nabla_\mu\nabla^\mu \phi = \partial_\mu \left(\sqrt{|g|} g^{\mu\nu} \partial_\mu \phi\right) = 0
\end{align*}

For solutions we also require to drop to zero at the boundary. Define a scalar product of two such solutions $\phi, \psi$ over a Cauchysurface \(\Sigma\) via:
\begin{align*}
(\phi|\psi) &:= i \int_{\Sigma} \mathrm{d}S^\mu\, \phi^*\nabla_\mu \psi - \psi\nabla_\mu \phi^* = i \int_{\Sigma} \mathrm{d}S^\mu\, \phi^*\overset{\leftrightarrow}{\nabla}_\mu \psi
\end{align*}

(Quelle: Townsend)The scalar product is independent of the choice of \(\Sigma\): Assume two Cauchysurfaces \(\Sigma\), \(\Sigma'\) and denote the 'sandwiched' region between them by \(A\). Then by Gauss' law
\begin{align*}
(\phi|\psi) - (\phi|\psi)' &= i\int_{\Sigma}\mathrm{d}S^\mu\, \phi^*\nabla_\mu \psi - \psi\nabla_\mu \phi^* - \int_{\Sigma'}\mathrm{d}S^\mu\, \phi^*\nabla_\mu \psi - \psi\nabla_\mu \phi^*\\
	&= i\int_{A} \sqrt{|g|} \mathrm{d^4}x\,\nabla^\mu \left(\phi^*\nabla_\mu \psi - \psi\nabla_\mu \phi^*\right)\\
	&= i\int_{A} \sqrt{|g|} \mathrm{d^4}x\,\phi^*\nabla^\mu\nabla_\mu \psi - \psi \nabla^\mu\nabla_\mu\phi^* = 0.
\end{align*}

Note that \((\phi^*|\psi^*) = -(\phi|\psi)^*\) and \((\phi|\psi)^* = (\psi|\phi)\).\\

Now choose a complete set of solutions $\{u_i\}$, such that
\begin{align*}
(u_i| u_j) = \delta_{ij},\,(u_i^*| u_j^*) = -\delta_{ij}\,\text{and}\,(u_i^*| u_j) = 0
\end{align*}

Using a Bra-Ket-like notation the completeness can be written by \(\mathbb{I} = \sum_i |u_i)(u_i| - |u_i^*)(u_i^*|\)

\section{Quantisation, Bogolubov Transformations and Vacua}

(Quelle: arXiv1011.5875 )We can quantize the field by introducing the canonical commutation relations CCR on a cauchysurface \(\Sigma\) with (future directed) normal vector $S^\mu$:

\begin{align*}
[\phi(x),\phi(x')]_\Sigma &= 0\\
[\phi(x),\nabla_S \phi(x')]_\Sigma &= i\delta(x-x')\\
[\nabla_S \phi(x),\nabla_S \phi(x')]_\Sigma &= 0\\
\end{align*}

One can show that if they hold on one cauchysurface they hold on every cauchysurface.\\
Given a complete set of modes this leads to \(\phi = \sum_i u_i a_i + u_i^* a_i^\dagger\), with \(a\) a bosonic annihilation operator satisfying \([a_i,a_j^\dagger] = \delta_{ij}\).\\

Of course there are many different complete sets. One could also expand it in a different set \(\{v_j\}\): \(\phi = \sum_j v_j b_j + v_j^* b_j^\dagger\). The b's are then given by \(b_j = \sum_i (v_j|u_i) a_i - (v_j|u_i*) a_i^\dagger\). This is called a Bogolubov transformation. 

TODO: Vacua

\section{Particle Detectors}
The treatment of a particle detector follows (Quelle: Birell Davies) and goes back to Unruh. One describes a detector by a operator \(m(\tau)\) which couples to the field via a interaction term \(c\cdot m(\tau) \phi(x(\tau))\), where \(c\) is small and \(x(\tau)\) is the trajectory of the detector. For \(\tau \to -\infty\) the detector is in the groundstate \(\ket{E_0}\) and the field is in the vacuum state \(\ket{0}\). The detector develops with time according to \(m(\tau) = e^{i H_0 \tau} m(0) e^{-i H_0 \tau}\) with \(H_0 \ket{E} = E\ket{E}\).\\
We would like to calculate the probability that the detector detects a particle with energy \(E\). Since \(c\) is small one can use first order perturbation theory where the transition amplitude to another state \(\ket{E,\psi}\) at time \(\tau\) is given by
\begin{align*}
A_{\ket{E_0,0}\to\ket{E,\psi}}(\tau) &= i c \bra{E,\psi} \int_{-\infty}^\tau m(\tau') \phi(x(\tau'))\mathrm{d}\tau'\ket{E_0,0}\\
	&= i c \bra{E,\psi} \int_{-\infty}^\tau e^{i H_0 \tau'} m(0) e^{-i H_0 \tau'} \phi(x(\tau'))\mathrm{d}\tau'\ket{E_0,0}\\
	&= i c \bra{\psi} \int_{-\infty}^\tau e^{i E \tau'} \bra{E}m(0)\ket{E_0}  e^{-i E_0 \tau'} \phi(x(\tau'))\mathrm{d}\tau'\ket{0}\\
	&= i c \bra{E}m(0)\ket{E_0} \int_{-\infty}^\tau e^{i (E-E_0) \tau'} \bra{\psi}\phi(x(\tau'))\ket{0}\mathrm{d}\tau'\\
\end{align*}

The transition probability is \(P_{\ket{E_0,0}\to\ket{E,\psi}}(\tau) = |A_{\ket{E_0,0}\to\ket{E,\psi}}(\tau)|^2\). But since we are only interested in the state of the detector we sum over all field configurations:
\begin{align*}
P_E(\tau) &:= \sum_{i} P_{\ket{E_0,0}\to\ket{E,\psi_i}}(\tau) = \sum_{i}  |A_{\ket{E_0,0}\to\ket{E,\psi}}(\tau)|^2\\
		  &= c^2 |\bra{E}m(0)\ket{E_0}|^2 F_{E-E_0}(\tau)\\
\text{with}\,F_E(\tau) &= \sum_{i}\left|\int_{-\infty}^\tau e^{i (E-E_0) \tau} \bra{\psi_i}\phi(x(\tau'))\ket{0}\mathrm{d}\tau'\right|^2\\
	&= \sum_{i} \int_{-\infty}^\tau e^{-i E \tau''} \bra{0}\phi(x(\tau''))\mathrm{d}\tau''\ket{\psi_i}\bra{\psi_i}\int_{-\infty}^\tau e^{i E \tau'} \phi(x(\tau'))\ket{0}\mathrm{d}\tau'\\
	&= \int_{-\infty}^\tau\mathrm{d}\tau' \int_{-\infty}^\tau \mathrm{d}\tau'' e^{-i E (\tau''-\tau')} \bra{0}\phi(x(\tau'')) \phi(x(\tau'))\ket{0} := \int_{-\infty}^\tau\mathrm{d}\tau' \int_{-\infty}^\tau \mathrm{d}\tau'' e^{-i E (\tau''-\tau')} D^+(x(\tau''), x(\tau'))
\end{align*}

There we introduced the Wightman function \(D^+(x,x') = \bra{0}\phi(x) \phi(x')\ket{0}\). The probability splits in a product of two parts. The first one only depends on the model of the detector while the second part only depends on the trajectory. We will therefore interpret the function \(F_E(\tau)\) as the distribution of energy excitations as been 'seen' by an observer on the trajectory \(x(\tau)\).\\
The transition rate is then given by:
\begin{align*}
\cfrac{\mathrm{d}F_E(\tau)}{\mathrm{d}\tau} &= \int_{-\infty}^\tau \mathrm{d}\tau'' e^{-i E (\tau''-\tau)} D^+(x(\tau''), x(\tau)) \phi(x(\tau))\ket{0} + \int_{-\infty}^\tau \mathrm{d}\tau' e^{-i E (\tau-\tau')} D^+(x(\tau), x(\tau'))\\
&= \int_{-\infty}^\tau \mathrm{d}\tau' e^{-i E (\tau'-\tau)} D^+(x(\tau'), x(\tau)) + e^{-i E (\tau-\tau')} D^+(x(\tau), x(\tau'))\\
&\overset{\tilde{\tau} = \tau'-\tau}{=} \int_{-\infty}^0 \mathrm{d}\tilde{\tau} e^{-i E \tilde{\tau}} D^+(x(\tilde{\tau} + \tau), x(\tau)) + e^{i E \tilde{\tau}} D^+(x(\tau), x(\tilde{\tau} + \tau))\\
&= 2 \mathrm{Re} \int_{-\infty}^0 \mathrm{d}\tilde{\tau} e^{-i E \tilde{\tau}} D^+(x(\tilde{\tau} + \tau), x(\tau))
\end{align*}

since \(D^+(x,x')* = D^+(x',x)\). For the special case that the Wightman function does not depend on \(\tau\), i.e. \(D^+(x(\tau_1 + \tau'),x(\tau_2 + \tau')) = D^+(x(\tau_1),x(\tau_2))\) one can simplify this further.

\begin{align*}
\cfrac{\mathrm{d}F_E(\tau)}{\mathrm{d}\tau} &=  \int_{-\infty}^0 \mathrm{d}\tilde{\tau} e^{-i E \tilde{\tau}} D^+(x(\tilde{\tau} + \tau), x(\tau)) + \int_{0}^\infty \mathrm{d}\tilde{\tau} e^{- i E \tilde{\tau}} D^+(x(\tau), x(\tau - \tilde{\tau}))\\
&= \int_{-\infty}^0 \mathrm{d}\tilde{\tau} e^{-i E \tilde{\tau}} D^+(x(\tilde{\tau} + \tau), x(\tau)) + \int_{0}^\infty \mathrm{d}\tilde{\tau} e^{- i E \tilde{\tau}} D^+(x(\tau  + \tilde{\tau}), x(\tau))\\
&= \int_{-\infty}^\infty \mathrm{d}\tilde{\tau} e^{-i E \tilde{\tau}} D^+(x(\tilde{\tau} + \tau), x(\tau)) = \int_{-\infty}^\infty \mathrm{d}\tilde{\tau} e^{-i E \tilde{\tau}} D^+(x(\tilde{\tau}), x(0))
\end{align*}

The rate is the fouriertransform of the Wightman function and is independent of \(\tau\).